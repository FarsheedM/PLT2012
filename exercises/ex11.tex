\documentclass[a4paper,12pt]{article}

% Use the Classic Thesis style
%
% \usepackage[nochapters]{classicthesis}

\usepackage[T1]{fontenc}
\usepackage[utf8]{inputenc}

% Use text fonts
%
\usepackage{lmodern}  % the Latin Modern fonts which is an enhanced version of the Computer Modern fonts
% \usepackage{tgschola}  % the Gyre Schola fonts which is an enhanced version of the New Centrury Schoolbook fonts

% Use math fonts
%
% \usepackage{concmath}  % the Concrete Math fonts
% \usepackage{euler}  % the Euler Math fonts
% \usepackage{kerkis}  % the Kerkis fonts

% Some font packages that provide math font may override the text font, the
% following three commands reset it to the default.
%
% For the Latin Modern fonts
%
% \renewcommand{\rmdefault}{lmr}
% \renewcommand{\sfdefault}{lmss}
% \renewcommand{\ttdefault}{lmtt}
%
% For the Gyre Schola fonts
%
% \renewcommand{\rmdefault}{qcs}

\usepackage{amsmath}

% Use the MH bundle
%
% \usepackage{mhsetup}
% \usepackage{mathtools}
% \usepackage{mathstyle}
% \usepackage{breqn}
% \usepackage{empheq}
% \usepackage{flexisym}

% Use AMS-style theorem
%
\usepackage{amsthm}
% \theoremstyle{definition}
% \newtheorem{theorem}{Theorem}

% Use MATH formulas in PARagraph mode Typesetting Inference Rules
% 
\usepackage{mathpartir}

% \usepackage{tikz}
% \usepackage{tikz-qtree}
% \usepackage{bigfoot}

% New commands
%
\newcommand{\appl}[2]{#1\inparens{#2}}
\newcommand{\eval}[2]{#1 \Longrightarrow #2}
\newcommand{\redc}[2]{#1 \longrightarrow #2}

\usepackage{stmaryrd}  % the St Mary’s Road symbol font


\newcommand{\atangles}[1]{\langle#1\rangle}
\newcommand{\atparens}[1]{(#1)}
\newcommand{\atbracks}[1]{[#1]}
\newcommand{\atbraces}[1]{\{#1\}}
\newcommand{\atfloors}[1]{\lfloor#1\rfloor}
\newcommand{\atceils}[1]{\lceil#1\rceil}
\newcommand{\atucorners}[1]{\ulcorner#1\urcorner}
\newcommand{\atlcorners}[1]{\llcorner#1\lrcorner}
\newcommand{\atlparens}[1]{\llparenthesis#1\rrparenthesis}
\newcommand{\atlbracks}[1]{\llbracket#1\rrbracket}
\newcommand{\atbars}[1]{|#1|}
\newcommand{\atbbars}[1]{\|#1\|}

\newcommand{\inangles}[1]{\langle#1\rangle}
\newcommand{\inparens}[1]{\left(#1\right)}
\newcommand{\inbracks}[1]{\left[#1\right]}
\newcommand{\inbraces}[1]{\left\{#1\right\}}
\newcommand{\infloors}[1]{\left\lfloor#1\right\rfloor}
\newcommand{\inceils}[1]{\left\lceil#1\right\rceil}
\newcommand{\inucorners}[1]{\left\ulcorner#1\right\urcorner}
\newcommand{\inlcorners}[1]{\left\llcorner#1\right\lrcorner}
\newcommand{\inlparens}[1]{\left\llparenthesis#1\right\rrparenthesis}
\newcommand{\inlbracks}[1]{\left\llbracket#1\right\rrbracket}
\newcommand{\inbars}[1]{\left|#1\left|}
\newcommand{\inbbars}[1]{\left\|#1\right\|}

\newcommand{\texterr}[1]{\textcolor{red}{#1}}

\newenvironment{grammar}
 {\begin{center}\small\begin{tabular}{rrclr}}
 {\end{tabular}\normalsize\end{center}}
\newcommand{\prodhead}[5]{#1 & #2 & #3 & #4 & #5 \\}
\newcommand{\prodrule}[3]{      & & #1 & #2 & #3 \\}


\newcommand{\rarr}{\rightarrow}
\newcommand{\lamb}{\lambda}

\newcommand{\NatT}{\textbf{Nat}}
\newcommand{\zero}{\textbf{0}}
\renewcommand{\succ}{\textbf{succ}}
\newcommand{\pred}{\textbf{pred}}
\newcommand{\isze}{\textbf{iszero}}

\newcommand{\BlnT}{\textbf{Bool}}
\newcommand{\true}{\textbf{true}}
\newcommand{\fals}{\textbf{false}}
\newcommand{\ifkw}{\textbf{if}}
\newcommand{\thkw}{\textbf{then}}
\newcommand{\elkw}{\textbf{else}}

\newcommand{\UniT}{\textbf{Unit}}
\newcommand{\unit}{\textbf{unit}}

\newcommand{\Pair}{\textbf{Pair}}
\newcommand{\pair}{\textbf{pair}}
\newcommand{\fstp}{\textbf{fst}}
\newcommand{\sndp}{\textbf{snd}}

\newcommand{\inlh}{\textbf{inl}}
\newcommand{\inrh}{\textbf{inr}}
\newcommand{\case}{\textbf{case}}
\newcommand{\ofkw}{\textbf{of}}
\newcommand{\Rarr}{\Rightarrow}

\newcommand{\letk}{\textbf{let}}
\newcommand{\inkw}{\textbf{in}}

\newcommand{\askw}{\textbf{as}}

\newcommand{\fixp}{\textbf{fix}}

\newcommand{\IntT}{\textbf{Int}}



% Front elements
%
\title{
 Programming Languages and Types \\~\\
 \textbf{Exercise 11}
}
\author{
 Yi Dai
}

\begin{document}

\maketitle


\section{$\lambda$-Calculus vs. $\lambda$-Based Language}

Many languages have claimed to be based on $\lambda$-calculus.  Being $\lambda$-calculus-based does
\emph{not} mean they include $\lambda$-calculus as a subset or they offer every feature $\lambda$-calculus
offers.  Actually there are quite some differences.
\begin{enumerate}
 \item $\lambda$-calculus uses \emph{static scoping}.  This sounds strange because nowadays almost all
  languages (whether functional or imperative) that supports functions (whether higher-order or just
  first-order), use static scoping.  However, this point deserves an emphasis because the first functional
  language that claimed to be $\lambda$-calculus, mamely \emph{Lisp}, ironically uses \emph{dynamic scoping}.
  This \emph{bug} or \emph{mis-feature} were not discovered for over two decades until Scheme came out.
  Then after over one decade, static scoping was finnally written into the Common Lisp ANSI standard.  What
  a historical lession!
 \item $\lambda$-calculus supports \emph{symbolic computation} while most $\lambda$-calculus-based languages
  do not.\footnote{Actually, most languages do not support symbolic computation.  Exceptions may be those
  languages that were designed for AI, such as Lisp and Prolog.  Though their support is via a new data type
  for symbols, which is kinda \emph{indirect}.}  Symbolic computation allows a free variable to compute to
  itself if there is no binding associated to it.  So in $\lambda$-calculus, the variable $x$ evaluates to
  itself.  Whereas in most functional languages, you get a run-time error complaining that the variable is
  \emph{unbound}.
 \item $\lambda$-calculus features full $\beta$-reduction.  In other words, evaluation goes under $lambda$.
  Whereas in most functional language, evaluation stops in front of a $lambda$.  Instead of a
  $lambda$-abstraction with a fully-reduced body, you get a closure.  Although we gain some efficiency, we
  also miss part of the beauty of $\lambda$-calculus.  For example, we cannot see how Church-numerals work.
\end{enumerate}

\section{Nameless $\lambda$-Calculus}

\subsection{$\alpha$-Conversion: Pros and Cons}

\subsection{$\lambda$-Calculus Using De-Bruijn Indices}

\subsection{Evaluation Semantics for $\lambda_{bv}$}

\section{Substitution vs. Environment}

\subsection{Substitution-Based Evaluation Semantics for $\lambda_{ao}$}

\subsection{Environment-Based Evaluation Semantics for $\lambda_{bn}$}

\section{Computational $\lambda$-Calculus}

\subsection{$\lambda$-Calculus Extended with Monads}

\subsection{Typing $\lambda_c$}

\end{document}
