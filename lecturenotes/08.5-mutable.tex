% latex
\documentclass{beamer}
\usepackage{ngerman}
\usepackage[utf8]{inputenc}
\usefonttheme{serif}\setbeamertemplate{navigation symbols}{}
\setbeamercolor{background canvas}{bg=}
\begin{document}

\begin{frame}
\frametitle{Transcript: advantages of mutation}
\vtop{\hsize=0.5\hsize
\begin{itemize}
\pause
\item Efficiency
\begin{itemize}
\item Space (esp. in call stack)
\item Time
\end{itemize}
\pause
\item Loops are easier to understand than recursion
\pause
\item Memoisation is difficult to express declaratively
\end{itemize}
}
\end{frame}

\begin{frame}
\frametitle{Transcript: disadvantages of mutation}
\begin{itemize}
\pause
\item Laziness causes problems
\pause
\item Von-Neumann bottleneck
\pause
\item Lack of referential integrity

(Wegen Reihenfolgeabhängigkeit)
\end{itemize}
\end{frame}

\begin{frame}
\frametitle{The book}
Peter Van Roy, Seif Haridi.

Concepts, Techniques, and Models of Computer Programming.

MIT Press, March 2004.

Section 4.8, page 313.
\end{frame}

\begin{frame}
\frametitle{Subjectivity}
Different people may find different models easier to use, because of
their differing knowledge and background. The issue is not the precise
definition of \emph{natural}, but the fact that such a definition
exists for each person, even though it might be different for
different people.
\end{frame}

\begin{frame}
\frametitle{John Backus's Turing Award lecture}
The assignment statement is the von Neumann bottleneck of programming
languages and keeps us thinking in word-at-a-time terms in much the
same way the computer's bottleneck does.
\end{frame}

\begin{frame}[fragile]
\frametitle{Edsger W. Dijkstra's review}
A naive implementation [of immutable matrices] that first copies those
matrices and then kicks out half of it again seems absolutely
unacceptable on any machine---von Neumann or not.

\pause
\begin{verbatim}
www.haskell.org/haskellwiki/GHC/Memory_Management
\end{verbatim}
\end{frame}

\begin{frame}
\frametitle{Naturalness versus efficiency}
A programme is \emph{natural} if very little code is needed just for
technical reasons unrelated to the problem at hand.
\end{frame}

\begin{frame}[fragile]
\frametitle{A natural Haskell programme}
\begin{verbatim}
fib 0 = 1
fib 1 = 1
fib n = fib (n - 2) + fib (n - 1)
\end{verbatim}
\end{frame}

\begin{frame}[fragile]
\frametitle{A fast Haskell programme}
\begin{verbatim}
fib2 n =
 let loop 0 prev next = next
     loop m prev next = loop (m - 1) next (prev + next)
 in
     loop n 0 1
\end{verbatim}
\end{frame}

\begin{frame}
\frametitle{A matter of compiler intelligence?}
It is unrealistic to expect the compiler to rewrite your programme.
Even after several decades of research, no such compiler exists for
general-purpose programming.
\end{frame}

\begin{frame}[fragile]
\frametitle{Functional means natural?}
\begin{verbatim}
case Letrec(x, e, body) => {
  val mutableenv = scala.collection.mutable.Map() ++ env
  mutableenv += x -> eval(e, mutableenv)
  eval(body,mutableenv)
}
\end{verbatim}
\end{frame}

\begin{frame}[fragile]
\frametitle{Functional means natural?}
Circular lists in Haskell

\begin{verbatim}
www.haskell.org/haskellwiki/Tying_the_Knot
\end{verbatim}
\end{frame}

\begin{frame}
\frametitle{Simplicity: where?}
The declarative model is one of the simplest of all. It has serious
limitations for some applications. There are more expressive models
that overcome these limitations, at the price of sometimes making
reasoning more complicated.
\end{frame}

\begin{frame}
\frametitle{Simplicity: where?}
The more expressive models are not \textit{better} than the others,
since they do not always give simple programmes and reasoning in them
is usually harder.
\end{frame}

\end{document}
